% \documentclass[12pt]{ociamthesis}  % default square logo
\documentclass[12pt]{article}  % default square logo

\usepackage[utf8]{inputenc}
\usepackage[T1]{fontenc}
\usepackage[english]{babel}

\usepackage[margin=2.8cm]{geometry}
\usepackage{setspace}
\usepackage{mathptmx} % Times font
\usepackage[utf8]{inputenc}
\usepackage{graphicx}
\usepackage{framed}
\usepackage[colorlinks=true]{hyperref}

% code listing
\usepackage{listings}
\usepackage{xcolor}

% packages for drawing
\usepackage{tikz}
\usetikzlibrary{graphs}     % create graphs
\usetikzlibrary{arrows}

%% set code styles

\definecolor{dkgreen}{rgb}{0,0.6,0}
\definecolor{gray}{rgb}{0.5,0.5,0.5}
\definecolor{mauve}{rgb}{0.58,0,0.82}

\lstset{frame=tb,
  language=scala,
  aboveskip=3mm,
  belowskip=3mm,
  showstringspaces=false,
  columns=flexible,
  basicstyle={\small\ttfamily},
  numbers=left,
  numberstyle=\tiny\color{gray},
  keywordstyle=\color{blue},
  commentstyle=\color{dkgreen},
  stringstyle=\color{mauve},
  breaklines=true,
  breakatwhitespace=true,
  tabsize=3,
}


\onehalfspacing{}

\title{Case study: the board game Go}

\author{Ólafur Páll Geirsson  \quad \quad  Fengyun Liu}

%end the preamble and start the document
\begin{document}

\maketitle

\begin{abstract}
    The ancient Chinese board game Go has fascinated computer scientists and mathematicians for its enormous state space.
    In this case study, we implemented Go in Leon and Scala with the objective to answer the question: how practical is it to use Leon to program a correct implementation of Go?
    Our experiments reveal that it is indeed useful to to annotate a Scala codebase with Leon verification conditions.
    Still, we consider that Leon has plenty of room for improvement to make it easier for a programmer to verify non-trivial properties about her programs.
\end{abstract}

% \tableofcontents

\section{Contributions}
The main contributions of this project are the following:
\begin{itemize}
    \item We found a productive developer workflow combining Leon verification and Scala
        unit testing (see Section~\ref{sec:Developer_workflow}).
    \item We did a minor case study of the connected component algorithm (see Section~\ref{sec:Connected_Component}).
    % \item We verified several properties of our implementation of the board game Go (see Section~\ref{sec:Verification}).
    \item We identified a few issues in Leon, which we reported on Github (see Section~\ref{sec:Bugs}).
\end{itemize}

\section{Developer workflow} % (fold)
\label{sec:Developer_workflow}
Although we wanted to verify certain properties in Leon, we set the requirement for our project that we also wanted to be able to run our Go program with Scala in the command line.
This requirement forced us to set up a productive developer workflow that executed both the Scala and Leon versions of our code.

A common developer workflow with Scala is to have an \emph{sbt} shell open inside your project and execute the command \texttt{$\sim$test}.
On every file save, sbt will then recompile your project and execute your unit tests.
The benefit to this workflow is that the code-compile-run-debug cycle is as short as possible.
When starting on our project, we quickly realized that we missed this productive workflow.

We were able to accomplish the same productive workflow with Leon by making a few minor changes to our \texttt{build.sbt} file.
The key idea was to make a new sbt task, which we named \emph{verify}.
The verify task was defined with the following snippet of code in our \texttt{build.sbt} definitions.
\lstinputlisting{code/build.sbt}
This snippet adds the bash script \texttt{leonVerify.sh} to the files that are watched by sbt.
The implementation of the task simply runs the unit tests and then if the all unit tests compile and pass, we run the Leon verification.
The contents of the \texttt{leonVerify.sh} script is then:
\lstinputlisting{code/leonVerify.sh}
Instead of running \texttt{$\sim$test} in the sbt shell, we now ran \texttt{$\sim$verify}.
Whenever we modified our code or tweaked the parameters supplied to Leon in the \texttt{leonVerify.sh} script, the task was re-executed.

This setup greatly helped us keep the codebase compatible with both Scala and Leon.
Although Leon aims to be a strict subset of Scala, there are a few minor discrepancies between the two langauges.
Such an example includes \texttt{Nil[T]()}, which compiles with Leon but does not compile with Scala since \texttt{scala.Nil} is a singleton type.
Moreover, there are many methods in Scala collection libraries that are yet to be added to the Leon collections.
Some additional benefits to using this developer workflow include:
\begin{itemize}
    \item Compilation errors are detected faster with scalac, due to sbt's incremental compilation.
    \item Obvious bugs are caught faster by the unit tests than verification.
    \item Not so obvious bugs in corner cases were detected by our unit tests while Leon timed out, see \href{https://github.com/olafurpg/sav-project/blob/621e03d97f510d516fee87e5cff0985c37b434cb/leon-go/src/main/scala/go/core/CaptureLogic.scala#L60}{here}.
    \item Obviously non-terminating programs failed the unit tests before verification was executed.
\end{itemize}
% section Developer workflow (end)

\section{Connected component} % (fold)
\label{sec:Connected_Component}
The connected component algorithm is useful to implement the capture rule of the game correctly.
Stones on the board are ``captured'' whenever themselves and no stone in their connected group of the same color has a ``liberty''.
If a cell is captured, it is removed from the board.
A stone has liberty if a neighbor cell is not occupied by another stone.
A simple and way to implement the capture rule is to identify the connected groups of stones in the board, and then filter out the stones that have no liberty.

One way to implement a connected component algorithm is to use a depth-first search.
The stones are the nodes and two stones have an edge between them if they are of the same color and lie next to each other on the board.
We pick a random stone to start our search and then visit the rest of the stone's component.
It turned out to be more complicated than we anticipated to verify the correctness of this algorithm.

We first implemented the algorithm with the following code:
\lstinputlisting{code/cc1.scala}
Our initial goal was to verify with the precondition
\begin{center}
    \texttt{board.isValidPoints(toVisit) \&\& board.isValidPoints(component)}
\end{center}
where \texttt{isValidPoints} says that the points lie inside perimeter of the board, then the method would only return valid points.
The method \texttt{sameColorNeighborPoints}, that we had verified, had a postcondition saying that it only returned points inside the board.
We were confident that Leon could verify that no points outside the board are introduced in the method.

Contrary to our expectations, Leon timed out.
After some experiments we figured out that the issue had to do with line 12, where we concatenate two lists.
Leon could not infer that if the \texttt{board.isValidPoints} holds for two lists, then it also holds for the concatenation of those lists.
We added the helper method \texttt{addValidElements}
\lstinputlisting{code/addValidElements.scala}
which we used in place of the \texttt{List.++} method.
Leon was able to verify our condition immediately.

Next, we were able to verify that

We tried made several failed attempts to verify the following properties of the algorithm:
\begin{itemize}
    \item \emph{All points were of same color.}

        Even though \texttt{sameColorNeighborPoints} had a post condition that
        all the points had the same color, Leon could not verify this property.
        This was perhaps our biggest disappointment in this case study.

    \item \emph{All points have a path to the root.}

        We implemented a \texttt{isConnected(u, v)}, which we used in our
        postcondition to verify that all points in the result were connected to
        the root.  We made several attempt in this direction but with no
        success.  We believe that difficulty lied in that fact that the
        relation from \texttt{isConnected} is transitive, which we struggled
        forward to Leon.

    \item \emph{The component is maximal}

        We tried to verify that there was no stone left on the board that was
        not part of the result but had a connection to the component.
        Given our poor results from the other verification attempts, we had
        little time to explore this condition further.
\end{itemize}

% section Connected Component (end)

\section{Github issues} % (fold)
\label{sec:Bugs}
We reported the following issues on Github:
\begin{itemize}
        \foreach\issue\ in {94,98,99,103,105,106,107}
    {\item \href{https://github.com/epfl-lara/leon/issues/\issue}{\#\issue}}
\end{itemize}

Please follow the links for more details.
% section Bugs (end)


\section{Conclusion}

During the project, we learned some lessons about Leon and verification in
general.

First, we find that \emph{class state invariants} are helpful, which is a nice
feature to have in Leon.  For example, in the class Point, we’d like very
instance of Point to satisfy the pre-condtion. Unfortunately, we have to
duplicate the requirement in almost every usage of Point.

\lstinputlisting{code/class-invariant.scala}

Class state invariant not only avoids duplicate code, but also enables
programmers to think about verification at a higher level of abstraction, and
it’s more natural with the object-oriented mind-set.

Second, we find that Leon can be improved to reduce surprises for developers.
Following are two examples.

\begin{itemize}
    \item Unit tests help against termination
\end{itemize}

\lstinputlisting{code/surprise.scala}

\begin{itemize}
    \item Publish leon dummies to Maven.
\end{itemize}

\end{document}

\documentclass[11pt]{article}
\usepackage[utf8]{inputenc}
\usepackage[T1]{fontenc}
% \usepackage{acl2012}
\usepackage{times}
\usepackage{latexsym}
\usepackage{amsmath}
\usepackage{multirow}
\usepackage{url}
\usepackage[english]{babel}
\usepackage{enumerate}
\usepackage{graphicx}
\usepackage{fancyhdr}
\usepackage{booktabs}
\usepackage{float}
\usepackage{hyperref}
\usepackage{nopageno}
\usepackage{listings}
\usepackage[norelsize,ruled,vlined]{algorithm2e}



%%%%%%%%%%%%%%%%%%%%%%%%%%%%%%%%%%%%%%%%%%%%%%%%%%%%%%%%%%%%%
%                        Setup
%%%%%%%%%%%%%%%%%%%%%%%%%%%%%%%%%%%%%%%%%%%%%%%%%%%%%%%%%%%%%


\begin{document}
\title{The board game Go in Leon}

\author{
    Liu Fengyun \\
    Ólafur Páll Geirsson
  }

\maketitle


\begin{abstract}
    The ancient Chinese board game Go has fascinated computer scientists and
    mathematicians for its enormous state space.  In this project, we propose
    to implement the rules of Go in Leon and verify certain properties of the
    game.  If time allows, we will try to integrate our game with Scala.js and
    make some interactive web-based user interface.
\end{abstract}

\section{Plan of attack}
Our plan of attack is as follows:
\begin{enumerate}
    \item Implement the game with a text-based ASCII user interface
    \item Verify certain properties, such as:
        \begin{itemize}
            \item A play cannot make an illegal move according to the rules of
                the game
            \item If a player respects the rules of the game, then no illegal
                states are reachable
            \item There are no runtime errors\footnote{We are not entirely sure
                how to accomplish this}
        \end{itemize}
    \item Synthesize several simple strategies for a computer player.
    \item If time allows, we would be excited to compile our Leon program with
        Scala.js and write an interactive web-based user interface with a
        library such as \emph{scalajs-react}.
\end{enumerate}

\end{document}




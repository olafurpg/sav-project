\documentclass[11pt]{article}
\usepackage[utf8]{inputenc}
\usepackage[T1]{fontenc}
% \usepackage{acl2012}
\usepackage{times}
\usepackage{latexsym}
\usepackage{amsmath}
\usepackage{multirow}
\usepackage{url}
\usepackage[english]{babel}
\usepackage{enumerate}
\usepackage{graphicx}
\usepackage{fancyhdr}
\usepackage{booktabs}
\usepackage{float}
\usepackage{hyperref}
\usepackage{nopageno}
\usepackage{listings}
\usepackage[norelsize,ruled,vlined]{algorithm2e}



%%%%%%%%%%%%%%%%%%%%%%%%%%%%%%%%%%%%%%%%%%%%%%%%%%%%%%%%%%%%%
%                        Setup
%%%%%%%%%%%%%%%%%%%%%%%%%%%%%%%%%%%%%%%%%%%%%%%%%%%%%%%%%%%%%


\begin{document}
\title{Scala.js integration with Leon to improve counterexample presentation in the web interface}

\author{
    Liu Fengyun \\
    Ólafur Páll Geirsson
  }

\date{}
\maketitle


\begin{abstract}
    Although a main feature of Leon is to provide counterexamples to invalid programs, it can remain difficult when provided with a counterexample to understand how the input breaks the program.
    We propose to integrate Scala.js with Leon and implement ideas from Bret Victor's essay on ``Learnable Programming''\footnote{See \url{http://worrydream.com/#!/LearnableProgramming}} to improve experience for users when presented with a counterexample.

\end{abstract}

\section{Foobar} % (fold)
\label{sec:Foobar}

% section Foobar (end)

\end{document}



